
%%% Template originaly created by Karol Kozioł (mail@karol-koziol.net) and modified for ShareLaTeX use

\documentclass[a4paper,11pt]{article}

\usepackage[T1]{fontenc}
\usepackage[utf8]{inputenc}
\usepackage{graphicx}
\usepackage{xcolor}
\usepackage{booktabs}
\usepackage{float}
\usepackage{graphicx}
\usepackage{pdflscape}

\usepackage{tgtermes}

\usepackage[
pdftitle={Course Project Discussion}, 
pdfauthor={Li Zhi, Wang Ran,University of Windsor},
colorlinks=true,linkcolor=blue,urlcolor=blue,citecolor=blue,bookmarks=true,
bookmarksopenlevel=2]{hyperref}
\usepackage{amsmath,amssymb,amsthm,textcomp}
\usepackage{enumerate}
\usepackage{multicol}
\usepackage{tikz}
\usepackage{gensymb}

%%%%%%%%%%%%%%%%%%%%%%%%%%%%%%%%%%%%%%%%%%%%%%%%%%%%%%%%%%%%%%%%%%%%%%%%%%%%%%%biaogexiangguan
\usepackage{array}
\makeatletter
\newcommand{\thickhline}{%
    \noalign {\ifnum 0=`}\fi \hrule height 1pt
    \futurelet \reserved@a \@xhline
}
\newcolumntype{"}{@{\hskip\tabcolsep\vrule width 1pt\hskip\tabcolsep}}
\makeatother


\usepackage{geometry}
\geometry{total={210mm,297mm},
left=25mm,right=25mm,%
bindingoffset=0mm, top=20mm,bottom=20mm}


\linespread{1.3}

\newcommand{\linia}{\rule{\linewidth}{2pt}}

% custom theorems if needed
\newtheoremstyle{mytheor}
    {1ex}{1ex}{\normalfont}{0pt}{\scshape}{.}{1ex}
    {{\thmname{#1 }}{\thmnumber{#2}}{\thmnote{ (#3)}}}

\theoremstyle{mytheor}
\newtheorem{defi}{Step}
\newtheorem{defi2}{Step}
\newtheorem{defithree}{}
\newtheorem{defifour}{}
\newtheorem{defifive}{}

% my own titles
\makeatletter
\renewcommand{\maketitle}{
\begin{center}
\vspace{2ex}
{\huge \textsc{\@title}}
\vspace{1ex}
\\
\linia\\
\@author \hfill \@date
\vspace{4ex}
\end{center}
}
\makeatother
%%%

% custom footers and headers
\usepackage{fancyhdr,lastpage}
\pagestyle{fancy}
\lhead{}
\chead{}
\rhead{}
\lfoot{Flow Measurements}
\cfoot{}
\rfoot{Page \thepage\ /\ \pageref*{LastPage}}
\renewcommand{\headrulewidth}{0pt}
\renewcommand{\footrulewidth}{0pt}
%

%%%----------%%%----------%%%----------%%%----------%%%
\begin{document}

\title{assignment 1}

\author{Name, \textnumero{}xxx-xxxx, University of XX}

\date{05/21/2016}

\maketitle
\section{Problem 1}
\begin{defi}
Find mean value
\end{defi}
\begin{equation}
    \begin{array}{rcl}
           \overline{d} & = &  \displaystyle\sum\limits_{i=1}^{10} \frac{d_i}{10} 
              =   49380 \quad kJ/kg
    \end{array}
\end{equation}
\begin{defi}
Find measurement bias limit
\end{defi}
Calorimeter does not introduce any precision error into the measurement, so instrument error is zero $U_c = 0$. Zero order uncertainty($U_0$) is equal to $2500/2 = 1250$ kJ/kg.
\begin{equation}
    \begin{array}{rcl}
           B & = &  \sqrt{U^2_0 + U^2_c} = 1250 \quad kJ/kg
    \end{array}
\end{equation}
\begin{defi}
Find corrected sample standard deviation:
\end{defi}
\begin{equation}
    \begin{array}{rcl}
           s_d & = &  \sqrt{\frac{\sum\limits_{i=1}^{10}(d_i-\bar{d})^2}{N-1}} = 537.07 \quad kJ/kg
    \end{array}
\end{equation}
Precision: 
\begin{equation}
    \begin{array}{rcl}
           \rho & = & \frac{s_d}{\sqrt{N}} = \frac{537.07}{\sqrt{10}} = 169.84 \quad kJ/kg
    \end{array}
\end{equation}
so the overall uncertainty is 169.84 kJ/kg if the mean value of the sample of measurements is taken as the heating value.
If a single measurement of heating value is to be taken,then 
\begin{equation}
    \begin{array}{rcl}
           U_{\bar{d}} & = & \sqrt{B^2+(t_{\mu,\rho} \rho)^2} = \sqrt{1250^2+(t_{\mu,\rho} 169.84)^2}= 1307.8 \quad kJ/kg
    \end{array}
\end{equation}
where, $ t = 2.265$.
The overall uncertainty is 1307.8 kJ/kg. 
\section{Problem 2}
\begin{defi2}
Find relative uncertainty values
\end{defi2}
\begin{equation}
    \begin{array}{rcl}
           \frac{u_{\mu}}{\mu} & = & \frac{0.02}{3.00}   =  0.67 \quad \% \\
           \frac{u_{V}}{V} & = & \frac{0.05}{1.23}       =  4.07 \quad \% \\
           \frac{u_{L}}{L} & = & \frac{0.02}{0.50}       =  4.00 \quad \% \\
           \frac{u_{B}}{B} & = & \frac{0.001}{0.02}      =  0.5 \quad \% \\
%           \frac{u_{\theta}}{\theta} & = & \frac{1}{30}  =  3.33 \quad \% 
    \end{array}
\end{equation}
Denote $N = sin \theta$, the relative uncertainty for N is: 
\begin{equation}
    \begin{array}{rcl}
          u  =  |cos \, \theta |u_x  &=&  |cos \, \frac{\pi}{6}|\frac{\pi}{180} = 0.015 \\    
          u_r   =  \frac{u}{sin \frac{\pi}{6}} & =&  \frac{0.015}{0.5} =3 \quad \%
%          \frac{u}{sin} & = & \frac{0.001}{0.02} = 3 \quad \%
    \end{array}
\end{equation}
\begin{defi2}
Find relative uncertainty values of mass
\end{defi2}
\begin{equation}
    \begin{array}{rcl}
          m  =  \frac{\mu V L^2}{Bgsin\theta} = \mu V L^2 g^{-1}B^{-1} N^{-1}    
    \end{array}
\end{equation}
where N = sin $\theta$.
\begin{equation}
    \begin{array}{rcl}
          \frac{u_m}{m}  = \pm \sqrt{(1 \times \frac{u_{\mu}}{\mu})^2+(1 \times \frac{u_V}{V})^2+(2 \times \frac{u_L}{L})^2+(-1\times \frac{u_B}{B})^2 + (-1 \times \frac{u_N}{N})^2 } = \pm 10.72 \quad \%  
    \end{array}
\end{equation}
The percentage relative uncertainty in the measured mass is $\pm 10.72 \%$.
\end{document}



%%% Local Variables:
%%% mode: latex
%%% TeX-master: t
%%% End:
